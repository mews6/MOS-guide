% Auth: Nicklas Vraa
% Docs: https://github.com/NicklasVraa/LiX

\documentclass{textbook}

\lang      {english}
\title     {Modeling, Simulation and Optimization}
\subtitle  {(For the Layman)}
\author    {Jaime Torres}
\cover*    {resources/textbook_front.pdf}{resources/textbook_back.pdf}
\license   {CC}{by-nc-sa}{3.0}{The Company}
\isbn      {978-0201529838}
\publisher {Nobodym just me!}
\edition   {1}{2024}
\dedicate  {Myself}{Because I'm cool}
\thank     {Thank you to me for being the best}
\keywords  {optimization, simulation, python, mathematical modeling}

\begin{document}

\tableofcontents

\chapter{Mathematical Modeling.}

For this chapter, we'll:

\begin{itemize}
    \item Get our framework from an algorithmic framework of thinking to algebraic 
    framework
    \item Represent a problem with algebraic expressions.
    \item Make proper implementations of mathematical models.
\end{itemize}

\section{Introduction to mathematical modeling}

Before we can begin to think about models, we must understand what they are.
And, for the purposes of this book, we can define a mathematical model as a representation
of a real life problem with an abstract representation of the phenomenons we intend to study.

For an immediately relevant example, a lot of problems in graph theory can be explained in algebraic terms.
Although these problems can be a bit more mathematical in nature, then we can use them to explain real-life problems.
Such as:

\begin{itemize}
    \item Shortest path in a graph, for computer networking and pathfinding.
    \item The traveling salesman problem, for pathfinding, DNA sequencing and microchip manufacturing 
    \item The vehicular routing problem, for finding multiple instances of a TSP-esque solution on a same graph
\end{itemize}

As a practical example of such a graph-based problem, let's analyze the following:

\subsection{Example 1: The transport problem}

Let a finite number of sources and a finite number of destinations, we can determine the
number of elements we can route to every destination with a minimal cost for every source. 
For example, if we said such sources are factories and the destinations are warehouses, we could see a 3x3
graph as it follows:

% \includegraphics[options]{name}

\subsection{Example 2: Cover problems}

More than a specific problem, this is a subdivisio of problems meant to cover a certain group of demands
or coverage criteria according to a specific defined problem.

\subsubsection{The hospital problem}

For example, say we got a group of neighborhoods that are of euclidian behavior, and we intend to put the minimum amount of
hospitals that can cover the entire region we're considering. A hospital covers the neighborhood is located on and
those who share a boundary with it. A boundary is a non-zero

For an example, imagine the following instance of this problem:

%% answer is 5 and 8

\subsubsection{The four color theorem}

The four color theorem is a very famous application of a cover problem. 

% \includegraphics[scale=0.4]{notes/resources/Map_of_United_States_accessible_colors_shown}

\subsection{Example 3: Minimum spanning tree}

If we imagine a tree data structure such as a graph with no cycles, and a graph as an interlocked set of
nodes, a minimum spanning tree would be the minimum cost path that can connect all nodes as a tree.

This problem has a set of requirements for it to be considered correct:

\begin{itemize}
    \item There must be no cycles or subcycles in the solution
    \item the number of connections must be the number of nodes minus one
\end{itemize}

\section{Classification of mathematical models}

These models can be, as most things in mathematics, categorized and calificated in 

\subsection{Deterministic mathematical models}

These are models that can be predicted in a way that is certain.

\subsection{Stochastic mathematicam models}

A stochastic model is one that involves a certain amount of randomness. For example, 

\subsection{Static mathematical models}

\subsection{Dynamic mathematical models}
\subsubsection{Continuous time models}
\subsubsection{Discrete time models}

A fairly interesting application of discrete time models comes from finance, where there exists 
capitalization in a set time interval. For example, we could imagine the free cash flow of an economic project as



\end{document}
